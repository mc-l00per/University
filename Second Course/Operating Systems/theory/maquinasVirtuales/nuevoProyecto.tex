\input{confgTex.tex}

%----------------------------------------------------------------------------------------
%	TÍTULO Y DATOS DEL ALUMNO
%----------------------------------------------------------------------------------------

\title{	
	\normalfont \normalsize 
	\textsc{\textbf{Sistemas Operativos (2017-2018)} \\ Grado en Ingeniería Informática \\ Universidad de Granada} \\ [25pt] % Your university, school and/or department name(s)
	\horrule{0.5pt} \\[0.4cm] % Thin top horizontal rule
	\huge Maquinas Virtuales \\ % The assignment title
	\horrule{2pt} \\[0.5cm] % Thick bottom horizontal rule
	\begin{figure}[H] %con el [H] le obligamos a situar aquí la figura
		\centering
		\includegraphics[scale=0.5]{image/ugr.png}  %el parámetro scale permite agrandar o achicar la imagen. En el nombre de archivo puede especificar directorios
	\end{figure}
}

\author{Antonio Rodríguez Alaminos} % Nombre y apellidos

\date{\normalsize\today} % Incluye la fecha actual

%----------------------------------------------------------------------------------------
% DOCUMENTO
%----------------------------------------------------------------------------------------

\begin{document}
	
	%----------------------------------------------------------------------------------------
	% PORTADA
	%----------------------------------------------------------------------------------------
	
	\maketitle % Muestra el Título
	
	\newpage %inserta un salto de página
	
	\tableofcontents % para generar el índice de contenidos
	
	%----------------------------------------------------------------------------------------
	% INDICE
	%----------------------------------------------------------------------------------------
	
	%\listoffigures
	
	%\listoftables
	
	\newpage
	
	%----------------------------------------------------------------------------------------
	% CUESTION 1
	%----------------------------------------------------------------------------------------
	
	\chapter{Introducción}
	
	\section{¿Qué es Virtualización?}
	
	Se podría definir Virtualización como la tecnología que permite ejecutar varios sistemas operativos o máquinas virtuales en una misma máquina física. Esta tecnología puede ser implementada por hardware o por software, siendo esta última más versátil, flexible y extendida. \cite{1} \newline
	
	En una computadora donde se virtualice van a coexistir dos tipos de sistemas operativos:
	
	\begin{itemize}
		\item {Sistema operativo anfitrión: Es el que manejará el hardware, el que deberá tener los controladores adecuados para interactuar con los componentes instalados. En este se definirá qué recursos serán asignados a cada una de las máquinas virtuales.}
		
		\item {Sistema operativo huésped: Son los sistemas operativos que correrán dentro de las máquinas virtuales definidas en el servidor. Cada uno de estos sistemas operativos interactúa con el hardware físico a través de dispositivos virtuales generado en una capa de virtualización gestionada por el sistema operativo anfitrión. Los recursos de hardware asignados a cada máquina virtual podrán ser modificados según las necesidades, asignándole más memoria, mayor almacenamiento, o bien, quitándole o agregándole interfaces de red. Estas modificaciones podrán realizarse con la máquina virtual apagada o encendida dependiendo de las capacidades de la solución de virtualización implementada.} \newline
		
	\end{itemize}
	
	Teniendo en cuenta que cada sistema operativo asume que tiene el control del hardware sobre el que corre, es evidente que para lograr que los sistemas operativos puedan convivir se requiere de un servidor de virtualización que administre el acceso a la memoria, el uso del CPU y por supuesto las entradas/salidas. Este servidor es una capa de software que interactúa con el sistema operativo anfitrión y con los sistemas operativos huéspedes.

	
	\section{Tipos de Virtualización}\nopagebreak[0]
	
	Los sistemas de virtualización se pueden clasificar entendiendo tres paradigmas distintos: los sistemas de virtualización completa (Full Virtualization), los sistemas de paravirtualización y los que se conocen como virtualización a nivel de sistemas operativo:
	
	\begin{itemize}
		\item Full Virtualization: Es el paradigma más utilizado hoy en día. Los sistemas operativos huéspedes desconocen estar siendo virtualizados, por lo tanto no requieren ser modificados. Se utiliza un hipervisor entre las máquinas virtuales y el hardware. Los sistemas operativos huéspedes realizan las peticiones directamente al hardware virtualizado y ante cualquier instrucción que requiera acceso privilegiado el hipervisor la intercepta para su apropiado manejo. Esta arquitectura de virtualización facilita la migración de máquinas virtuales entre diferentes sistemas, e incluso entre diferentes arquitecturas. Esta flexibilidad y versatilidad tiene aparejado una penalización en la performance y el uso de recursos. Ejemplos: \newline
			
			\begin{itemize}
				\item {VMWare}
				\item {Qemu-KVM}
				\item {VitualBox}
			\end{itemize}
			
		
		\item Paravirtualización: Esta tecnología implica una modificación en el sistema operativo huésped que incorpore cambios en los llamados a drivers evitando la necesidad del hipervisor de hacer de proxy entre el sistema operativo huésped y el hardware. En estos sistemas cada máquina virtual actúa con el hardware de manera similar a cualquier otro proceso del sistema. Ejemplos: 
			
			\begin{itemize}
				\item {Xen}
				\item {LDoms}
			\end{itemize}
			
			
		\item Virtualización a nivel de Sistema Operativo: Este sistema es muy diferente a los dos anteriores. Este tipo de virtualización permite a un sistema operativo crear múltiples entornos de aplicaciones aislados que referencian al mismo kernel. Una ventaja importante que tienen es que su penalización es muy baja. Ejemplos: AIX partitions, Solaris containers. \cite{3}
		
	\end{itemize}
	
		
		
	\section{Procesos y Modos de ejecución}
		
	Un proceso es un programa en ejecución. Tradicionalmente, un proceso en Linux tiene dos modos de ejecución: kernel y usuario. El modo usuario (user mode) es un modo de ejecución sin privilegios comúnmente usado por aplicaciones de usuario, o sea cualquier código que no pertenezca al kernel. \newline
	
	El modo kernel (kernel mode) es solicitado cuando una aplicación requiere servicios del kernel, como puede ser escribir a disco. Ante la necesidad de ejecutar una operación de E/S o crear un nuevo proceso hijo, el proceso en modo usuario debe realizar una llamada al sistema para solicitar la acción que requiere ejecutar. 
		
	\section{Tipos de control de accesos}
		
	El control de accesos es un mecanismos de protección interno que tienen los sistemas operativos multiusuario para proteger los recursos de accesos indebidos. \newline
	
	Unix/Linux históricamente ha usado el esquema de control de acceso discrecional (DAC). Donde generalmente el propietario del recurso o archivo en cuestión, administra los derechos de acceso sobre el mismo y también puede especificar a cual de sus grupos pertenece el recurso. \cite{6} \newline
	
	Sumado a esto, debemos saber que un proceso en ejecución toma los derechos de acceso que tiene el usuario que lo ha invocado. Así un proceso puede modificar o borrar un archivo en particular, si el usuario que lo invoca cuenta con los derechos de acceso suficientes en el recurso. \cite{5} \newline
	
	Siguiendo esta lógica es que el kernel decide si un proceso puede o no leer, escribir o ejecutar un recurso en particular. Aunque debemos saber también que esta lógica es usada para usuarios normales, el super usuario root no sigue esta lógica, ya que es el dueño del sistema operativo y el tiene acceso completo a todos los recursos del sistema. \newline
	
	En contraste con lo anterior existe el control de acceso mandatorio (MAC), implementado en sistemas open sources a través del kernel SELinux. En este tipo de control de acceso cada objeto (archivos o procesos) tiene una etiqueta que representa el nivel de sensibilidad de la información que contiene. Adicionalmente cada usuario tiene asociado un nivel de permisos que indica a que tipos de objetos puede acceder. Estas autorizaciones son especificadas en las políticas SELinux determinado ademas que permisos tendrá al hacerlo. \cite{5} \newline
		
	\newpage
	\chapter{Seguridad.}
	\section{Seguridad en Virtualización Open Source}
	
	La seguridad en virtualización debe ser estudiada desde un punto de vista un poco más amplio que al pensarlo para un servidor sin virtualización. Una brecha de seguridad en una máquina virtual podría traer problemas en otras máquinas virtuales del mismo host físico o incluso podría comprometer el mismo host anfitrión. Así mismo si el host anfitrión es inseguro todas las máquinas virtuales serán vulnerables. \newline
	
	Cuando se analiza la seguridad de un hipervisor se deben tener en cuenta los riesgos introducidos por la plataforma de virtualización, que generalmente estarán relacionados con:
	
	\begin{itemize}
		\item El nivel de aislamiento entre máquinas virtuales, tanto a nivel de red como de entrada/salida; 
		\item La protección de los accesos a las consolas de las máquinas virtuales, antes física, ahora remota; 
		\item El aseguramiento de los accesos a la gestión y administración del hipervisor.
	\end{itemize}

	El presente análisis y sus recomendaciones han sido pensados para escenarios cada vez más comunes en la actualidad, en los que un servidor de virtualización alberga servidores de diferentes zonas de seguridad, por ejemplo: servidores de DMZ, servidores de Intranet y servidores que atienden a Extranets de Partners todos corriendo dentro de un mismo servidor físico. \newline
	
	Como se puede prever la convivencia de diferentes zonas de seguridad aumenta la complejidad de administración y gestión del servidor de virtualización. Por este motivo, todas las configuraciones para aumentar la seguridad deben ser planificadas, analizadas, implementadas y por supuesto documentadas. La documentación de todos los esquemas y configuraciones resultan elementales para una administración cómoda y eficiente. \newline
	
	Además vale destacar que todas las recomendaciones y configuraciones son pensadas para usarse en Linux Debian Squeeze, aunque estas puede ser adaptadas para usarse en la mayoría de las
	distribuciones Linux. \newline
	
	Siendo tan amplio y disperso los temas que conciernen al análisis, para abarcar la problemática de seguridad en virtualización en su conjunto, esta se la analizará desde diferentes ángulos:
	\begin{itemize}
		\item La red: se analizará la interacción de las máquinas virtuales y el host anfitrión con el mundo. Se debe tener en cuenta que la proximidad y apareamiento de las máquinas virtuales entre sí, y con el host anfitrión es muy grande. Por lo tanto, el host anfitrión debería ser configurado de modo de aislar las máquinas virtuales de su propia red.
		\item El almacenamiento: se analizará el manejo de los datos y programas almacenados en medios físicos que usará cada máquina virtual y la propia del host anfitrión. Se debe asegurar el aislamiento de estos datos definiendo exactamente qué puede acceder cada proceso de kvm y restringiendo lo demás, para evitar posibles accesos indebidos.
		\item El acceso de administración: la creación, modificación y baja de hardware virtual, así como el encendido, migración y apagado de estas máquinas virtuales es una tarea de administración que debe ser asegurada mediante procesos de autentificación y autorización que impidan posibles ataques al hipervisor.
		\item El acceso a la consola de las máquinas virtuales: Debido a que las terminales de las máquinas huéspedes son consolas virtuales, se debe estudiar la manera mas conveniente de como restringir y controlar el acceso remoto a estas interfaces.
	\end{itemize}

	\section{Asegurando la red}
	
	A continuación se detallarán todos los temas relacionados con la disponibilidad, integridad y confidencialidad de los datos que viajan por la red en un host anfitrión y por supuestos en sus huéspedes.
	
	\section{Alta disponibilidad de red}
	
	Al incrementar la densidad de servicios que dependen de una misma interfaz de red, esta comienza a ser más crítica y se recomienda tomar ciertos recaudos. Esto sucede en la interfaz de red que brinda conectividad a las máquinas virtuales que corren dentro de un mismo host, el aumento de la criticidad de esta interfaz amenaza la disponibilidad de red para todas máquinas virtuales que se albergan. Por este motivo, es recomendable disponer de dos interfaces de red dedicadas para brindar la conexión a todas las máquinas dentro de un mismo equipo. \newline
	Para mayor facilidad de administración y una respuesta inmediata de alta disponibilidad lo conveniente es “unir” las interfaces bajo una virtual de modo que esta brinde servicio siempre que, al menos una de las interfaces físicas pueda hacerlo. Esta funcionalidad en linux se llama bonding y para el presente esquema el modo que usaremos es active-backup de modo que sólo una de las interfaces físicas atenderá en un momento y la otra será su backup. \newline
	Adicionalmente, es recomendable disponer de una tercera interfaz de red conectada a la red de Management o Administración que será utilizada exclusivamente para gestionar la máquinas virtuales de manera remota.
	
	\section{Más de una zona de seguridad virtualizada}
	
	En lo relativo a aislar diferentes zonas de seguridad partiremos del supuesto que la separación de estas, a nivel de red, lo tenemos implementado mediante VLANs. Partimos de esta base debido a que esta configuración es la más común en la actualidad. \newline
	Lo primero que debemos hacer es que el servidor de virtualización vea el conjunto de las VLANs de los zonas de seguridad tagueadas en protocolo IEEE 802.1q en ambas interfaces de servicio (eth0 y eth1). Para esto se deben configurar el switch al que están conectadas las interfaces para que en estas bocas se presenten las VLANs correspondientes. \newline
	
	Evitando ataques basados en técnicas de Spoofing ARP En caso de que una máquina virtual sea comprometida, esta podría ser utilizada para hacer un escalamiento horizontal del ataque. Muchos ataques se basan en el cambio de identidad de host atacante alterando la MAC address de la interfaz de red. Esta técnica es conocida como Spoofing ARP. \newline
	Para evitar esto lo que se recomienda hacer es definir reglas de firewall a nivel de capa física (capa 2 del modelo OSI). El objetivo de estas reglas será filtrar todo los paquetes que provengan de una interfaz virtual cuya MAC address no corresponda con la asignada al definir la máquina virtual. \newline
	
	\section{Las máquinas virtuales como procesos}
	Dan Walsh, líder de desarrollo de SELinux en Red Hat, afirma que correr múltiples sistemas operativos en un mismo hardware “es una de las cosas que más miedo dan” desde el punto de vista de la seguridad informática\cite{7}. Previo a la virtualización un host podía ser atacado a través de la red, pero si ahora varios sistemas corren sobre un mismo hardware, la “superficie de contacto” podría ser mucho mayor. \newline
	Como se detalla al comienzo del trabajo, Linux ve y trata a cada máquina virtual como un proceso más, este tiene asociado los privilegios del usuario que lo invocó. Ahora, supongamos que se descubre un nuevo bug que posibilita la ejecución de código en el hipervisor desde una máquina virtual. Solamente pensar en las consecuencias que esto podría tener da escalofríos.
	Al respecto se pueden tomar algunas medidas que se desarrollan a continuación:
	
	\begin{itemize}
		\item Un usuario no privilegiado por cada máquina virtual
		\item sVirt el módulo de SELinux
	\end{itemize}
	
\section{sVirt el módulo de SELinux}
	El servicio sVirt es un modulo incluido por SELinux para aislar la
	máquinas virtuales entre sí y brindar un esquema de control de acceso
	mandatorio (MAC) \cite{4}. Con esta funcionalidad se definen etiquetas únicas,
	de manera dinámica, para cada proceso que sea una máquina virtual
reduciendo los accesos autorizados para cada máquina sólo a sus recursos
asignados. De este modo, si una máquina virtual es comprometida, aún
	pudiendo escalar por sobre la virtualización para ejecutar código en el host
anfitrión, las posibilidades de explotar esta situación serán nula.
\newline
	
	Lamentablemente esta funcionalidad aún no está madura, ajustada y
pulida en Debian Linux, ya que las pruebas realizadas no han sido exitosas.
\newline
	
	Los inconvenientes encontrados al activar SELinux fueron:
	\begin{itemize}
		\item Presenta un error de falta de permisos al intentar asociar la interfaz túnel al bridge, esto sucede aun iniciando la máquina virtual con el
		usuario root.
 
		\item Luego de iniciada la máquina virtual (sin red por el problema anterior)
los discos y el proceso en cuestión, no son asignados con las
		etiquetas que deberían.
	\end{itemize}
	
	Contrastando los resultados obtenidos con los manuales de Red Hat
sobre el tema, es evidente que el desarrollo liderado por Dan Walsh no ha	sido compartido en su totalidad con la comunidad o esta no la ha
incorporado aún a los repositorios públicos.
Encriptando la unidad de almacenamiento de los huéspedes.
	La encripción del disco es una técnica que se suele usar para
proteger los datos almacenados en una unidad cuando esta no está en uso o
cuando el dispositivo está apagado. Es muy común la encripción de discos
en dispositivos portátiles para proteger la confidencialidad de los datos ante
un posible extravío.
\newline
	
	En el escenario de un hipervisor y sus máquinas virtuales esta técnica
podemos usarla para proteger los datos de las máquinas virtuales mientras
estas estén apagadas. Supongamos que un atacante logra acceder al
hipervisor, si la máquina virtual está apagada los datos de la misma se
encontrarán encriptados y por lo tanto inaccesibles. Estos tipos de ataques
son conocidos como “Ataques offline” \cite{4}.
	
	\section{Administración remota del hipervisor}
	El acceso remoto para administración del hipervisor debe ser
analizado en detalle, ya que es un punto crítico dentro de cualquier esquema
	de virtualización. Su criticidad radica en que, en caso de ser descuidado, un
acceso no autorizado amenazaría toda la infraestructura virtualizada.
\newline
	
	Dentro de los posibles esquemas de seguridad para el acceso de
administración y gestión de máquinas virtuales se detallarán tres formas de
	acceso remoto \cite{4}:
	
	\begin{itemize}
		\item Usando túneles SSH
		\item Usando autentificación SASL y encripción
		\item Usando TLS con llaves asimétricas
	\end{itemize}
	
	\section{Administración remota a través de túneles SSH}
	Este modo de acceso remoto aprovecha la seguridad implementada
en el protocolo de Secure Shell (SSH). Utiliza la conexión SSH como un
	túnel, que asegura integridad y confidencialidad dentro de la red, para
comunicarse con el hipervisor de manera remota.
	
	\section{Administración remota usando autentificación SASL y encripción}
	El protocolo SASL es un estándar usado por múltiples aplicaciones
que provee autentificación y encripción en conexiones. El servicio define un
	archivo de base de datos con las credenciales autorizadas. Además puede
ser usado en escenarios más complejos usando servicios de autentificación
externa como Kerberos o LDAP \cite{4}
	
	\section{Administración remota usando TLS con llaves asimétricas}
	Las conexiones TLS (Transport Layer Security) basan su seguridad en
encripción basada en certificados x509. Estos certificados son claves
	asimétricas firmadas por una autoridad certificantes (CA).
\newline
	
	Para generar los certificados existen diversas formas, la más básica
usando directamente el comando openssl. Pero para evitar que los
	certificados queden en claro en el disco usaremos las aplicaciones modutil y
certutil que brindan un entorno para la generación y almacenado de
	certificados de manera segura.
	
	\section{Administración remota de la consola de las máquinas virtuales}
	Un punto importante en la virtualización es que desaparecen las
consolas puras de las máquinas virtualizadas y estas pasan a ser virtuales
	también. Qemu-KVM brinda estas interfaces sobre el protocolo VNC para
acceso remoto. Por defecto esta opción está deshabilitada y deberá
	estudiarse en detalle previo a publicar el video de las máquinas virtuales. 
\newline
	
	El acceso a la consola remota deberá ser protegido de dos maneras:
exigiendo una autentificación mediante contraseña y habilitando el acceso al
	este puerto únicamente desde la estaciones de trabajo de los
administradores.
	
	\section{Protegiendo el acceso con contraseña}
	Como primera medida para asegurar el acceso a la consola de las
máquinas virtuales disponemos del uso de una contraseña en el servidor
	VNC que brinda acceso a dichas terminales. Aunque esto es seguro no es la
solución óptima, ya que la limitación del largo de las contraseñas es 8 (ocho)
caracteres haciendo que esta contraseña sea potencialmente vulnerable a ataques de diccionario. \cite{4} Por este motivo debemos elegir una contraseña
fuerte que incluya mayúsculas, minúsculas, números y caracteres
especiales. 
	
	\section{Protegiendo el acceso usando TLS y claves asimétricas}
	Para habilitar el acceso remoto vía VNC se recomienda hacerlo de
modo que este brinde autentificación TLS con certificados x509. Para
	configurar esto podemos reutilizar los certificados que hicimos para el
acceso a la administración del hipervisor. En caso de que quienes
	administren las máquinas virtuales y el hipervisor sean personas diferentes
será importante que los certificados usados sean diferentes. Para generar
nuevos certificados se puede seguir los pasos 1 al 6 de creación de	certificados explicado en el apartado “Administración remota usando TLS
con llaves asimétricas”.
	
	\chapter{Conclusión}
	
	Como se detalló a lo largo del trabajo, las variantes de configuración
	posibles son muchísimas. Esto exige a los administradores de este tipo de
	hipervisores un compromiso mayor con la seguridad ya que, como se vio, un
	error en la configuración podría comprometer la confidencialidad, integridad
	y disponibilidad de los datos y servicios alojados en el hipervisor y en las
	máquinas virtuales. \newline
	
	También es imprescindible conocer los nuevos riesgos
	introducidos con KVM para poder minimizarlos o llevarlos a un nivel
	aceptable para nuestra infraestructura.
\newline
	
	Aunque las herramientas a nivel de red y administración remota están
	preparadas para dar un soporte acorde a las necesidades de seguridad
	actuales, quedo evidenciado en las pruebas realizadas que todavía a nivel
	de procesos y privilegios de usuario del hipervisor hay muchos interrogantes
	por responder. \newline
	
	 Según la teoría y los manuales de Red Hat Enterprise,
	SELinux brinda una solución importante a través del modulo sVirt, que daría
	las respuestas pendientes en Debian Linux. De cualquier modo, un
	desarrollo más a conciencia y sin grandes cambios del módulo de
	administración de las máquinas virtuales puede permitirnos correr cada
	máquina virtual con un usuario diferente y de ese modo obtener un
	aislamiento más que aceptable sin caer en un modelo de control de acceso
	mandatorio. \newline
	
	Sin dudas es en este punto dónde aún queda mucho camino por
	recorrer, el presente trabajo no es más que un puntapié para un análisis más
	en profundidad sobre distintas alternativas de control de acceso y
	aislamiento de procesos que ayuden a minimizar los nuevos riesgos.
\newline
	
	El sistema de virtualización estudiado es muy “joven” y está en pleno
	desarrollo, hoy en día impulsado por ser el sistema de virtualización
	recientemente elegido por Red Hat, desplazando su anterior solución Xen.
	Debido a esto la comunidad que desarrolla Open Source posiblemente será
	altamente beneficiada del código que pueda brindar Red Hat.
\newline
	
	Más allá de los esfuerzos de los desarrolladores de la comunidad y de
	Red Hat, la responsabilidad de velar por la seguridad de los servidores que
	administremos es nuestra. Por este motivo, es muy importante mantener los
	hipervisores actualizados al día y seguir atentamente los reportes de
	seguridad. En este caso, DSA (Debian Security Announce) es el más
	indicado y en el que encontraremos todas las novedades al respecto.
\newline
	
	Por experiencia propia puedo asegurar que además de las
	configuraciones iniciales recomendadas a lo largo del trabajo, para
	maximizar la seguridad es fundamental la correcta documentación y revisión
	de los procedimientos para adaptarlos a este nuevo paradigma.
		
	%----------------------------------------------------------------------------------------
	%	Bibliografía
	%----------------------------------------------------------------------------------------
	
	%\footnote{Este es un ejemplo} comentario a pie de pagina
	
	%palabra \cite{mrx05,prueba2}. realizacion de referencias
	
	%\url dirección referenciar una url
	
	
	%para que no falle la bibliografia hay que generar el fichero bibliografia.bib y tendran el siguiente formato dentro
	%@article{mrx05, 
	%auTHor = "Mr. X", 
	%Title = {Something Great}, 
	%publisher = "nob" # "ody", 
	%YEAR = 2005, 
	%} 
	
	
	%------------------------------------------------
	
	\bibliography{bibliografia} %archivo citas.bib que contiene las entradas 
	\bibliographystyle{plain} % hay varias formas de citar
		
	
\end{document}


