%Pre�mbulo

\documentclass[paper=a4, fontsize=11pt]{scrartcl}

\usepackage[T1]{fontenc}
\usepackage[utf8]{inputenc}


\usepackage[spanish, es-tabla]{babel} 

\usepackage{amsmath,amsfonts,amsthm} 
\usepackage{graphics,graphicx, float} 



\usepackage{fancyhdr}
\pagestyle{fancyplain} 
\fancyhead{}
\fancyfoot[L]{} 
\fancyfoot[C]{} 
\fancyfoot[R]{\thepage} 
\renewcommand{\headrulewidth}{0pt} 
\renewcommand{\footrulewidth}{0pt} 
\setlength{\headheight}{13.6pt} 

\numberwithin{equation}{section} 
\numberwithin{figure}{section} 
\numberwithin{table}{section} 

\setlength\parindent{0pt} 

\newcommand{\horrule}[1]{\rule{\linewidth}{#1}} 

\title{	
\normalfont \normalsize 
\textsc{{\bf Disenio y Desarrollo de Sistemas Informaticos (2015-2016)} \\ Grado en Ingenieria Informatica \\ Universidad de Granada} \\ [25pt] 
\horrule{0.5pt} \\[0.4cm] 
\huge Requisitos funcionales, de informacion y restricciones semanticas \\ 
\horrule{2pt} \\[0.5cm] 
}

\author{Alejandro Duran Castro \\ Carmen Apellidos \\ Mohammed Apellidos \\ Antonio Apellidos} 

\date{\normalsize\today} 



\begin{document}

\maketitle 

\newpage

\tableofcontens

\newpage

\section{Introduccion}

	En este documento recoge de manera resumida los requisitos funcionales, de informacion y restricciones semanticas sobre el proyecto que desarrollaremos a lo largo del cuatrimestre en la asignatura.

\subsection{Presentacion del problema}

	Se desea desarrollar un software que permita la gestion de una comunidad de vecinos. Dicha comunidad puede estar compuesta por los inquilinos de un bloque de pisos, urbanizaciones, comunidades de adosados y similares.
\\[10pt]
	Entre las distintas necesidades que nuestro software debe solventar se encuentra la gestion de pagos de diversa indole. Estos podrian ser gastos de renta, alquiler de servicios, derramas comunitarias, perjucios de los inmueble.
\\[10pt]
	El software debe permitir gestionar el alquiler o reserva de los espacios privados de la comunidad como pistas de tenis, piscinas, cocheras, y el resto de servicios de que disponga.
\\[10pt]
	Para llevar a cabo estas tareas el sistema estara dotado de un registro de los pisos o casas. Tambien se alamacenara cierta informacion sobre los inquilinos.

\subsection{Funcionalidades del sistema}

Aqui ponemos a grandes rasgos las 4 funcionalidades de las que tendremos que sacar 4 RF.

\section{Requisitos funcionales}

Lista de requisitos con su especificacion


\section {Resquisitos de informacion}

Datos de clientes, casas, pistas, coches, plaza, etc.

\section {Restricciones semanticas}

no se pueden registrar camiones en el aparcamiento por problemas de espacio por poner un ejemplo

\end{document}
