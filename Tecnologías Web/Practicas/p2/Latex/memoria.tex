\input{confgTex.tex}

%----------------------------------------------------------------------------------------
%	TÍTULO Y DATOS DEL ALUMNO
%----------------------------------------------------------------------------------------

\title{	
\normalfont \normalsize 
\textsc{\textbf{Sistemas Multidimensionales (2016-2017)} \\ Grado en Ingeniería Informática \\ Universidad de Granada} \\ [25pt] % Your university, school and/or department name(s)
\horrule{0.5pt} \\[0.4cm] % Thin top horizontal rule
\huge Trabajo de Fin de Grado \\ % The assignment title
\horrule{2pt} \\[0.5cm] % Thick bottom horizontal rule
\begin{figure}[H] %con el [H] le obligamos a situar aquí la figura
	\centering
	\includegraphics[scale=0.5]{image/ugr.png}  %el parámetro scale permite agrandar o achicar la imagen. En el nombre de archivo puede especificar directorios
\end{figure}
}

\author{Antonio Rodríguez Alaminos \\ Yurena del Peso Pérez} % Nombre y apellidos

\date{\normalsize\today} % Incluye la fecha actual

%----------------------------------------------------------------------------------------
% DOCUMENTO
%----------------------------------------------------------------------------------------

\begin{document}

%----------------------------------------------------------------------------------------
% PORTADA
%----------------------------------------------------------------------------------------

\maketitle % Muestra el Título

\newpage %inserta un salto de página

\tableofcontents % para generar el índice de contenidos

%----------------------------------------------------------------------------------------
% INDICE
%----------------------------------------------------------------------------------------

\listoffigures

%\listoftables

\newpage

%----------------------------------------------------------------------------------------
% INICIO
%----------------------------------------------------------------------------------------

S

%----------------------------------------------------------------------------------------
%	Bibliografía
%----------------------------------------------------------------------------------------

%\footnote{Este es un ejemplo} comentario a pie de pagina

%palabra \cite{mrx05,prueba2}. realizacion de referencias

%\url dirección referenciar una url


%para que no falle la bibliografia hay que generar el fichero bibliografia.bib y tendran el siguiente formato dentro
	%@article{mrx05, 
	%auTHor = "Mr. X", 
	%Title = {Something Great}, 
	%publisher = "nob" # "ody", 
	%YEAR = 2005, 
	%} 


%------------------------------------------------

%\bibliography{bibliografia} %archivo citas.bib que contiene las entradas 
%\bibliographystyle{plain} % hay varias formas de citar

\end{document}
